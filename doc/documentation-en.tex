\documentclass[english]{article}

\usepackage[utf8]{inputenc}
\usepackage[T1]{fontenc}
\usepackage{natbib}
\usepackage{babel}
\usepackage{xcolor}
\usepackage{fontawesome5}
\usepackage{enumitem}
\usepackage{metalogo}
\usepackage{framed}
\usepackage{changes}
\usepackage{hyperref}

\setlength{\parskip}{1ex}

\definecolor{shadecolor}{rgb}{0.93,0.97,0.99}

\hypersetup{%
	breaklinks=true,%
	colorlinks=true,%
	allcolors=blue
}

\newlist{repertoires}{itemize}{2}
\setlist[repertoires]{label=\faIcon[regular]{folder-open}~}
\newlist{fichiers}{itemize}{1}
\setlist[fichiers]{label=\faIcon[regular]{file}~}

\newcommand{\cmd}[1]{%
	\texttt{\textbackslash#1\{\}}
}
\newcommand{\dec}[1]{%
	\texttt{\textbackslash#1}
}

\newcommand{\lien}[2]{%
	\href{#1}{#2 \faIcon{external-link-alt}}
}

\title{Journal Article Template for \emph{L'Actualité économique}}
\author{Benoit Hamel \\ Library, HEC Montréal}
\date{\today}

\definechangesauthor[color=red, name={Benoit Hamel}]{BH}

\begin{document}
	\maketitle
	
	\begin{abstract}
		The goal of this documentation is to explain the package's features to those who will use these
		templates. It is divided in two parts, depending on a person's role in the writing process.
	\end{abstract}
	
	\tableofcontents
		
	\section{For authors}
		\label{sec:auteurs}		
		
		\subsection{Files included in the package}
		
			On opening the package's archive file, you'll find the following structure:
			
			\begin{repertoires}
				\item \textbf{actuecon}
				\begin{repertoires}
					\item \textbf{doc}: documentation directory
					\item \textbf{img}: graphics and images directory
				\end{repertoires}
				\begin{fichiers}
					\item \textbf{actuecon.cls}: document class file
					\item \textbf{bibliographie.bib}: references file
					\item \textbf{econometrica.bst}: english bibliography style
					\item \textbf{econometrica-fr.bst}: french bibliography style
					\item \textbf{template-en.tex}: english template file
					\item \textbf{template-fr.tex}: french template file
				\end{fichiers}
			\end{repertoires}
		
			Using this package is pretty straightforward:
			
			\begin{itemize}
				\item You write your article in your language's template file 
					(\textbf{template-en.tex} or \textbf{template-fr.tex});
				\item Your write your bibliographic references in \textbf{Bib\TeX} format in the
					\textbf{bibliographie.bib} file;
				\item You save all your graphics and images in the \textbf{img/} directory.
			\end{itemize}
	
		\subsection{How to use the template file}
		
			This section explains the template file's usage, from the first to (almost) the last line.
			
			\subsubsection{The document class}
			
				The template file uses the \textbf{actuecon} document class, as you can see it in the
				\cmd{documentclass} command:
				
				\begin{shaded*}
					\verb|\documentclass[10pt,twoside,fleqn,french,english]{actuecon}|
				\end{shaded*}
			
				This class, based on the \textbf{article} class, was built to fully comply with the journal's
				presentation standards. \textbf{You must not modify the options} in the \cmd{documentclass}
				command nor in the \textbf{actuecon.cls} document class file.
				
			\subsubsection{The class' required packages}
			
				In the template's preamble, you will find the list of all required packages needed to use
				the \textbf{actuecon} document class and the templates. These packages are
				\textbf{already loaded} in the class file. There is no need to load them in the template file.
				
				When using packages for your article's purpose, please make sure that there is no
				compatibility issues with the class' packages.
				
			\subsubsection{Article metadata}
			
				The only metadata that you'll need to enter in the template file are the title
				(\cmd{AEtitre} command), the authors' list (\cmd{author} command)
				and their affiliation to an institution (\cmd{affil} command).
				
				You can add as many authors and affiliations as is necessary, as long as you alternate
				between \cmd{author} and \cmd{affil} commands.
				
				All the other article's metadata (volume, issue number, publication date and revisers) have
				to be entered by the journal's staff.
				
			\subsubsection{The article's body}
			
				The article's different sections are clearly defined using comments. Write your article
				while following these guidelines:
				
				\begin{itemize}
					\item Write your french and english \textbf{abstracts} in the \texttt{AEresume} and
						\texttt{AEabstract} environments.
					\item Write your \textbf{introduction}, your \textbf{main content} and your \textbf{conclusion} 
						in their respective section, making sure not to delete the \dec{AEintroduction},
						\dec{AEsectionsDeveloppement}, \dec{AEconclusion} and \dec{AEbibliographie} commands.
					\item In your article's sectioning hierarchy, do not go beyond the subsubsection.
				\end{itemize}
			
			\subsubsection{Bibliography}
			
				The package comes with two bibliography style files : \textbf{econometrica.bst} and 
				\textbf{econometrica-fr.bst}. These correspond to the english and french versions
				of the bibliography style used by the journal for references. You have to use either one
				of them, depending of the default language of your article.
	
	\section{For the journal's staff}
		\label{sec:personnel}
		
		This section explains what the journal staff has to do in order to work on an author's draft.
		
		\subsection{The document class}
		
			An ``editorial'' choice was made to separate the content from the container in the journal's
			article templates. Probabilities being high that an author will add his own packages, commands
			and environments, it was important to separate his customizations from the journal's in order
			to preserve the article's layout.
			
			This is why the \textbf{actuecon} document class was built : all of the article's layout is there.
			
			An author will only have to provide his template and references files, as well as his \texttt{img} folder
			as source code. You will be able to compile an article with a ``vanilla'' version of the class, and as such,
			you will make sure that the article meets all the presentation standards.
			
		\subsection{Article metadata}
		
			The draft's author having already entered the article's title, authors' list and affiliations, all you have
			to enter is the metadata related to the journal issue number.
			
			In the template's preamble, you'll find the \cmd{AEvolume} and \cmd{AEnumero} commands in which you must enter
			the volume and issue number of the journal to be published.
			
			The \cmd{AEdateParution} is used to enter the journal's \textbf{publication date}. It doesn't
			require a particular date format. It is just a plain text ``field'' in which you can enter a date as you see fit
			(``march-june 2017'', for example).
			
			The \cmd{date} command \textbf{must always stay empty} if you don't wan't another date to show up 
			in the title section -- try it out by inserting a date in the command to see what it does if you don't
			leave it blank.
			
			In the \textbf{document body}, you must enter the article's \textbf{first page} number in the
			\cmd{setcounter} command, as you can see it in the following example.
			
			\begin{shaded*}
				\begin{verbatim}
					% Article beginning at page 3.
					\setcounter{page}{3}
				\end{verbatim}
			\end{shaded*}
			
			This is necessary due to the fact that it is pretty difficult to merge many \LaTeX\ documents written
			by as many authors into one document without running into compatibility issues. At the end of the
			revision process, each document must be recompiled with its definitive first page number.
			
		\subsection{Draft revision}
		
			Revisers must first enter their name and initials in the \cmd{AEreviseur} command found in the draft's
			preamble, as you can see it in the following example. There can be as many revisers as necessary as
			long as there is a \cmd{AEreviseur} command for each one.
			
			\begin{shaded*}
				\verb|\AEreviseur{Benoit Hamel}{BH}|
			\end{shaded*}
		
			The revision process is made possible with the \LaTeX\ \emph{changes} package. It was made to
			copy Microsoft Word's behaviour in document revision. You can read the package's
			\lien{http://mirrors.ctan.org/macros/latex/contrib/changes/changes.english.pdf}{whole documentation}
			if you wish.
			
			In this documentation, we'll only look at three commands:
			
			\begin{itemize}
				\item \dec{added} to add text;
				\item \dec{deleted} to delete text;
				\item \dec{replaced} to replace text.
			\end{itemize}
		
			\subsubsection{Adding text}
			
				The \cmd{added} command has the following syntax:
				
				\begin{shaded*}
					\verb|\added[id=<initials>, remark=<remark>]{new text}|
				\end{shaded*}
			
				Reviser initials and the remark are optional. However, initials are useful for a reviser's
				identification in the case where there are more than one -- a draft's author can himself
				be a ``reviser''. Remarks must be short. The text to be added must be written between the
				curly braces, like in the following example.
				
				\begin{shaded*}
					\begin{verbatim}
						Ad aliquet amet commodo convallis dictum dignissim eu facilisis 
						faucibus fermentum hendrerit himenaeos inceptos massa ornare 
						purus quis risus sapien senectus taciti tempor torquent turpis 
						ultrices varius velit vestibulum.
						\added[id=BH]{Alea jacta est!}
					\end{verbatim}
				\end{shaded*}
			
				\begin{leftbar}
					Ad aliquet amet commodo convallis dictum dignissim eu facilisis 
					faucibus fermentum hendrerit himenaeos inceptos massa ornare 
					purus quis risus sapien senectus taciti tempor torquent turpis 
					ultrices varius velit vestibulum.
					\added[id=BH]{Alea jacta est!}
				\end{leftbar}
			
			\subsubsection{Deleting text}
			
				The \cmd{deleted} command has the following syntax:
				
				\begin{shaded*}
					\begin{verbatim}
						\deleted[id=<initials>, remark=<remark>]{deleted text}
					\end{verbatim}
				\end{shaded*}
			
				For the command to work properly, you must cut and paste the text that will be deleted
				in the curly braces, like in the following example.
				
				\begin{shaded*}
					\begin{verbatim}
						The \deleted[id=BH]{failing} @nytimes set \deleted[id=BH]{%
						liddle'} Bob Corker up by recording his conversation. 
						Was \deleted[id=BH]{made to sound} a fool and that's what 
						I am dealing with!
					\end{verbatim}
				\end{shaded*}
			
				\begin{leftbar}
					The \deleted[id=BH]{failing} @nytimes set \deleted[id=BH]{liddle'} Bob Corker 
					up by recording his conversation. 
					Was \deleted[id=BH]{made to sound} a fool and that's what 
					I am dealing with!
				\end{leftbar}
			
			\subsubsection{Replacing text}
			
				The \cmd{replaced} command has the following syntax:
				
				\begin{shaded*}
					\begin{verbatim}
						\replaced[id=<initials>, remark=<remark>]{new text}{%
						    old text}
					\end{verbatim}
				\end{shaded*}
			
				Like the \cmd{deleted} command, you must cut and paste the text to be replaced in
				the second pair of curly braces and write the new text in the first pair. The
				following example shows how to use the command.
				
				\begin{shaded*}
					\begin{verbatim}
						Democrat congresswoman totally \replaced[id=BH]{quoted}{%
						fabricated} what I said to the wife of a soldier who died
						in action (and \replaced[id=BH]{she has}{I have} proof).
					\end{verbatim}
				\end{shaded*}
			
				\begin{leftbar}
					Democrat congresswoman totally \replaced[id=BH]{quoted}{%
					fabricated} what I said to the wife of a soldier who died
					in action (and \replaced[id=BH]{she has}{I have} proof).
				\end{leftbar}
			
\end{document}