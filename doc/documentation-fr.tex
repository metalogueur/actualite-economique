%
% documentation-fr.tex
% Copyright 2018 HEC Montréal
%
% This work may be distributed and/or modified under the
% conditions of the LaTeX Project Public License, either version 1.3c
% of this license or (at your option) any later version.
% The latest version of this license is in
% https://www.latex-project.org/lppl/lppl-1-3c.txt
% and version 1.3c or later is part of all distributions of LaTeX
% version 2008/05/04 or later.
%
% This work has the LPPL maintenance status `maintained'.
% 
% The Current Maintainer of this work is Benoit Hamel
% <benoit.2.hamel@hec.ca>.
%
% This work consists of the files actuecon.cls, template-en.tex,
% template-fr.tex, documentation-en.tex, documentation-fr.tex
% and the derived files documentation-en.pdf and documentation-fr.pdf.
%
\documentclass[french]{article}

\usepackage[utf8]{inputenc}
\usepackage[T1]{fontenc}
\usepackage{natbib}
\usepackage{babel}
\usepackage{xcolor}
\usepackage{fontawesome5}
\usepackage{enumitem}
\usepackage{metalogo}
\usepackage{framed}
\usepackage{changes}
\usepackage{hyperref}

\setlength{\parskip}{1ex}

\definecolor{shadecolor}{rgb}{0.93,0.97,0.99}

\frenchbsetup{%
	og=«,fg=»,
	ItemLabels=\textbullet
}

\hypersetup{%
	breaklinks=true,%
	colorlinks=true,%
	allcolors=blue
}

\newlist{repertoires}{itemize}{2}
\setlist[repertoires]{label=\faIcon[regular]{folder-open}~}
\newlist{fichiers}{itemize}{1}
\setlist[fichiers]{label=\faIcon[regular]{file}~}

\newcommand{\cmd}[1]{%
	\texttt{\textbackslash#1\{\}}
}
\newcommand{\dec}[1]{%
	\texttt{\textbackslash#1}
}

\newcommand{\lien}[2]{%
	\href{#1}{#2 \faIcon{external-link-alt}}
}

\title{Gabarit pour les articles de la revue \emph{L'Actualité économique}}
\author{Benoit Hamel \\ Bibliothèque, HEC Montréal}
\date{\today}

\definechangesauthor[color=red, name={Benoit Hamel}]{BH}

\begin{document}
	\maketitle
	
	\begin{abstract}
		Cette documentation a pour but de présenter à tous ceux et celles qui auront à se servir du gabarit
		les fonctionnalités comprises dans ce paquetage. Elle se divise en deux parties, en fonction du
		niveau d'intervention de la personne dans le processus de rédaction.
	\end{abstract}
	
	\tableofcontents
		
	\section{Pour les auteurs}
		\label{sec:auteurs}		
		
		\subsection{Fichiers contenus dans le paquetage}
		
			À l'ouverture de l'archive, vous trouverez la structure suivante:
			
			\begin{repertoires}
				\item \textbf{actuecon}
				\begin{repertoires}
					\item \textbf{doc}: répertoire contenant la documentation
					\item \textbf{img}: répertoire pour l'insertion des graphiques et images
				\end{repertoires}
				\begin{fichiers}
					\item \textbf{actuecon.cls}: fichier de la classe de document
					\item \textbf{bibliographie.bib}: fichier des références bibliographiques
					\item \textbf{econometrica.bst}: style bibliographique anglais
					\item \textbf{econometrica-fr.bst}: style bibliographique français
					\item \textbf{template-en.tex}: gabarit anglais
					\item \textbf{template-fr.tex}: gabarit français
				\end{fichiers}
			\end{repertoires}
		
			L'utilisation du paquetage est fort simple:
			
			\begin{itemize}
				\item Vous rédigez votre article dans le fichier gabarit correspondant à la langue de rédaction (\textbf{template-en.tex} ou \textbf{template-fr.tex});
				\item Vous insérez vos références bibliographiques en format \textbf{Bib\TeX} dans le fichier \textbf{bibliographie.bib};
				\item Vous insérez tous vos graphiques et images dans le répertoire \textbf{img/}.
			\end{itemize}
	
		\subsection{Utilisation du fichier gabarit}
		
			La présente section explique en détails le gabarit, de la première à la (presque) dernière ligne.
			
			\subsubsection{La classe de document}
			
				Le gabarit utilise la classe de document \textbf{actuecon}, tel que vous pouvez le voir dans la
				commande \cmd{documentclass}:
				
				\begin{shaded*}
					\verb|\documentclass[10pt,twoside,fleqn,english,french]{actuecon}|
				\end{shaded*}
			
				Cette classe, dérivée de la classe \textbf{article}, a été conçue pour respecter intégralement les
				règles de présentation de la revue. Il est donc impératif que \textbf{vous ne modifiiez pas les options} de
				la commande ci-dessus ni le contenu du fichier de class \textbf{actuecon.cls}.
				
			\subsubsection{Paquetages requis par la classe}
			
				Dans le préambule du gabarit, vous trouverez la liste des paquetages requis pour l'utilisation de
				la classe de document \textbf{actuecon} et du gabarit. Ces paquetages sont \textbf{déjà chargés} dans
				le fichier de classe. Il est donc inutile de les charger à nouveau dans le gabarit.
				
				Lorsque vous chargez d'autres paquetages pour vos besoins personnels, assu-rez-vous que ceux-ci soient
				compatibles avec ceux de la classe.
				
			\subsubsection{Métadonnées de l'article}
			
				Les seules métadonnées que vous avez besoin de renseigner dans le gabarit sont le \textbf{titre} (commande
				\cmd{AEtitre}) et la \textbf{liste des auteurs} (\cmd{author}) et de leur \textbf{affiliation} à une institution (commande
				\cmd{affil}).
				
				Vous pouvez insérer autant d'auteurs et d'affiliations que nécessaire en vous assurant d'alterner les
				commandes \cmd{author} et \cmd{affil}.
				
				Les autres métadonnées de l'article (volume, numéro, date de parution et noms des réviseurs) doivent
				être saisies par le personnel de la revue.
				
			\subsubsection{Corps de l'article}
			
				Les différentes sections de l'article sont clairement identifiées à l'aide de commentaires. Rédigez
				votre article en respectant les directives suivantes:
				
				\begin{itemize}
					\item Rédigez vos \textbf{résumés}  français et anglais dans les environnements \texttt{AEresume} et
						\texttt{AEabstract} respectivement, sans laisser d'espaces entre les commandes \cmd{begin} et 
						\cmd{end} et votre texte de résumé.
					\item Rédigez votre \textbf{introduction}, votre \textbf{développement}, votre \textbf{conclusion} 
						et vos annexes dans les sections prévues
						à cet effet en veillant à ne pas supprimer les commandes \dec{AEintroduction},
						\dec{AEsectionsDeveloppement}, \dec{AEconclusion}, \dec{AEannexe} et \dec{AEbibliographie}.
					\item Dans la hiérarchie du sectionnement de l'article, ne descendez pas plus bas que la sous-sous-section.
				\end{itemize}
			
			\subsubsection{Bibliographie}
			
				Le paquetage vient avec deux fichiers de styles bibliographiques : \textbf{econometrica.bst} et 
				\textbf{econometrica-fr.bst}. Ceux-ci correspondent aux versions anglaise et française du style utilisé
				par la revue pour la mise en forme des références bibliographiques. Vous devez donc obligatoirement utiliser
				l'un ou l'autre des styles en fonction de la langue de rédaction.
				
		\subsection{Particularités éditoriales à prendre en compte}
		
			\subsubsection{Lettres grecques en mode mathématiques}
			
			Selon les spécifications du Studio de design graphique de HEC Montréal, les lettres grecques doivent
			être écrites en lettres droites et non pas italiques. Pour répondre à cette exigence, la classe de
			document \textbf{actuecon} utilise le paquetage \lien{http://mirrors.ctan.org/macros/latex/contrib/was/upgreek.pdf}{upgreek}.
			Veuillez consulter la documentation du paquetage afin de remplacer les commandes traditionnelles
			des lettres grecques par celles de \textbf{upgreek}.
	
	\section{Pour le personnel de la revue}
		\label{sec:personnel}
		
		La présente section explique ce que le personnel de la revue doit faire afin de finaliser le manuscrit soumis
		par un auteur.
		
		\subsection{La classe de document}
		
			Un choix «éditorial» a été fait de séparer le contenu du contenant pour le gabarit de la revue. Les probabilités
			étant très grandes qu'un auteur ajoute ses propres paquetages, commandes et environnements, il importait de
			séparer les personnalisations de celui-ci de celles de la revue, et ce, afin d'assurer la conservation de la mise en page de l'article.
			
			C'est pourquoi la classe de document \textbf{actuecon} a été créée : toute la mise en page de l'article s'y trouve.
			
			Un auteur n'aura donc qu'à fournir son fichier gabarit, son fichier de références et son dossier \texttt{img}
			contenant ses images et graphiques comme fichiers sources et vous pourrez compiler son article à partir d'une
			version «vanille» de la classe. Ce sera le moyen le plus simple de s'assurer que la mise en forme de l'article
			respecte les règles.
			
		\subsection{Métadonnées de l'article}
		
			L'auteur du manuscrit ayant déjà fourni le titre et la liste des auteurs et affiliations de l'article, il ne vous
			restera plus qu'à renseigner les métadonnées relatives au numéro de la revue dans lequel l'article sera publié.
			
			Dans le \textbf{préambule} du document principal, vous trouverez les commandes \cmd{AEvolume} et \cmd{AEnumero} dans
			lesquelles vous devrez renseigner le \textbf{volume} et le \textbf{numéro} de la revue à paraître.
			
			La \textbf{date de parution} de la revue doit être inscrite dans la commande \cmd{AEdateParution}. Cette commande ne requiert
			pas de
			format de date particulière; il s'agit plutôt d'un «champ» en texte libre. Vous pouvez donc inscrire la date comme
			vous l'entendez («mars-juin 2017», par exemple). D'après les spécifications du Studio de design graphique, vous devez
			\textbf{écrire le mois de parution en lettres minuscules}.
			
			La commande \cmd{date} doit \textbf{toujours rester vide}, et ce, afin d'empêcher l'affichage d'une date dans l'espace du
			titre de l'article -- testez-le en inscrivant une date dans la commande pour voir ce que ça donnera si vous ne laissez
			pas la commande vide.
			
			Dans le \textbf{corps du document}, vous devez inscrire le numéro de la \textbf{première page} de 
			l'article à	l'intérieur de la commande \cmd{setcounter}, comme dans l'exemple ci-dessous.
			
			\begin{shaded*}
				\begin{verbatim}
					% Exemple d'article commençant à la page 3.
					\setcounter{page}{3}
				\end{verbatim}
			\end{shaded*}
			
			Cela est nécessaire compte tenu du fait qu'il est très difficile de fusionner plusieurs documents \LaTeX\ rédigés par autant
			d'auteurs différents dans un seul document sans que des problèmes de compatibilité surviennent. À la toute fin du processus de révision, 
			chaque document devra donc être recompilé avec son numéro de première page définitive.
			
		\subsection{Révision d'un manuscrit}
		
			Les réviseurs doivent d'abord renseigner leur \textbf{nom} et leurs \textbf{initiales} dans la commande
			\cmd{AEreviseur} dans le préambule du manuscrit, comme dans l'exemple ci-dessous. Vous pouvez ajouter autant
			de réviseurs qu'il est nécessaire en ajoutant une commande \cmd{AEreviseur} pour chacun.
			
			\begin{shaded*}
				\verb|\AEreviseur{Benoit Hamel}{BH}|
			\end{shaded*}
		
			La révision se fait grâce au paquetage \LaTeX\ \emph{changes}. Il a été conçu pour copier le comportement de Microsoft Word dans la même situation. Vous pouvez lire la
			\lien{http://mirrors.ctan.org/macros/latex/contrib/changes/changes.english.pdf}{documentation complète} du
			paquetage si vous le désirez. 
			
			Dans le cadre de cette documentation, nous nous intéresserons à seulement trois commandes :
			
			\begin{itemize}
				\item \dec{added} pour ajouter du texte;
				\item \dec{deleted} pour supprimer du texte;
				\item \dec{replaced} pour remplacer du texte.
			\end{itemize}
		
			\subsubsection{Ajouter du texte}
			
				La commande \cmd{added} prend la forme suivante :
				
				\begin{shaded*}
					\verb|\added[id=<initiales>, remark=<remarque>]{nouveau texte}|
				\end{shaded*}
			
				Les initiales du réviseur et la remarque sont facultatives. Cependant, les initiales servent à l'identification
				de l'intervenant dans le cas où il y a plusieurs réviseurs -- l'auteur du manuscrit peut d'ailleurs s'ajouter en
				tant que «réviseur». La remarque, quant à elle, doit être courte. Le texte à ajouter est inscrit entre les accolades, comme dans l'exemple ci-dessous.
				
				\begin{shaded*}
					\begin{verbatim}
						Ad aliquet amet commodo convallis dictum dignissim eu facilisis 
						faucibus fermentum hendrerit himenaeos inceptos massa ornare 
						purus quis risus sapien senectus taciti tempor torquent turpis 
						ultrices varius velit vestibulum.
						\added[id=BH]{Alea jacta est!}
					\end{verbatim}
				\end{shaded*}
			
				\begin{leftbar}
					Ad aliquet amet commodo convallis dictum dignissim eu facilisis 
					faucibus fermentum hendrerit himenaeos inceptos massa ornare 
					purus quis risus sapien senectus taciti tempor torquent turpis 
					ultrices varius velit vestibulum.
					\added[id=BH]{Alea jacta est!}
				\end{leftbar}
			
			\subsubsection{Supprimer du texte}
			
				La commande \cmd{deleted} prend la forme suivante :
				
				\begin{shaded*}
					\begin{verbatim}
						\deleted[id=<initiales>, remark=<remarque>]{texte supprimé}
					\end{verbatim}
				\end{shaded*}
			
				Pour que la commande fonctionne convenablement, vous devez couper-coller le texte à supprimer
				dans celle-ci, comme dans l'exemple ci-dessous:
				
				\begin{shaded*}
					\begin{verbatim}
						The \deleted[id=BH]{failing} @nytimes set \deleted[id=BH]{%
						liddle'} Bob Corker up by recording his conversation. 
						Was \deleted[id=BH]{made to sound} a fool and that's what 
						I am dealing with!
					\end{verbatim}
				\end{shaded*}
			
				\begin{leftbar}
					The \deleted[id=BH]{failing} @nytimes set \deleted[id=BH]{liddle'} Bob Corker 
					up by recording his conversation. 
					Was \deleted[id=BH]{made to sound} a fool and that's what 
					I am dealing with!
				\end{leftbar}
			
			\subsubsection{Remplacer du texte}
			
				La commande \cmd{replaced} prend la forme suivante :
				
				\begin{shaded*}
					\begin{verbatim}
						\replaced[id=<initiales>, remark=<remarque>]{nouveau texte}{%
						    ancien texte}
					\end{verbatim}
				\end{shaded*}
			
				Tout comme la commande \cmd{deleted}, il faut couper-coller le texte à remplacer
				dans la deuxième accolade et inscrire le nouveau texte dans la première. L'exemple
				ci-dessous montre le résultat de son utilisation :
				
				\begin{shaded*}
					\begin{verbatim}
						Democrat congresswoman totally \replaced[id=BH]{quoted}{%
						fabricated} what I said to the wife of a soldier who died
						in action (and \replaced[id=BH]{she has}{I have} proof).
					\end{verbatim}
				\end{shaded*}
			
				\begin{leftbar}
					Democrat congresswoman totally \replaced[id=BH]{quoted}{%
					fabricated} what I said to the wife of a soldier who died
					in action (and \replaced[id=BH]{she has}{I have} proof).
				\end{leftbar}
			
		\subsection{Particularités éditoriales à prendre en compte}
		
			Les particularités suivantes sont toutes dictées par le Studio de design graphique.
			Il importe donc de les suivre afin d'accélérer le processus de publication.
		
			\subsubsection{Entêtes des pages d'un article}
			
			Le titre d'un article ne peut pas s'étendre sur plus d'une ligne. Lorsque cela se produit,
			il faut \textbf{tronquer le titre court} de la commande \cmd{AEtitre} afin qu'il se limite
			à une ligne.
			
			\subsubsection{Équations en blocs à l'intérieur d'une phrase}
			
			Il peut arriver qu'un auteur rédige une équation en bloc (c'est-à-dire à l'intérieur d'un
			environnement) alors que son équation se trouve en plein milieu d'une phrase. Cela a pour effet
			d'isoler l'équation dans un bloc et de couper le paragraphe et la phrase en deux.
			
			\LaTeX\ considérant qu'un nouveau paragraphe commence à la suite de l'équation, il indentera
			la première ligne du «nouveau» paragraphe. \textbf{Cette ligne ne doit pas être indentée}.
			Pour retirer l'indentation de la ligne, il suffit de la précéder de la commande \dec{noindent}.
			
			\subsubsection{Casse des noms des auteurs d'un article}
		
			Le nom de famille des auteurs d'un article doit être affiché en lettres majuscules dans le bloc de titre
			de l'article. Si les auteurs ont inscrit leur nom de famille avec la seule première lettre en majuscule,
			il importe que vous modifiiez la casse du nom vous-mêmes.
			
			\subsubsection{Césure dans les noms propres des citations}
		
			Lorsqu'une citation est insérée dans le texte, les noms propres ne doivent pas être coupés à
			la fin d'une ligne. Si un tel cas se présente, vous devez insérer la citation dans la
			commande \cmd{nohyphens} du package \textbf{hyphenat} chargé dans le fichier de la classe. La
			citation sera alors reformatée sans césure.
			
			\subsubsection{Police de caractères des liens URL}
			
			Certains auteurs utiliseront la commande \cmd{url} pour insérer des liens URL dans leur document.
			Cette commande affiche les liens avec une police à \texttt{largeur fixe}. Les liens devant
			conserver la même police que le reste du texte, vous devrez substituer la commande
			\verb|\href{URL}{text}| à \cmd{url}.
			
\end{document}